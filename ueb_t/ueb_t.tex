\documentclass[a4paper, 10pt, fleqn]{article}

\usepackage{layout}

\title{TKOM - Übertragungstechnik}
\author{daniw}
\date{\today}

\begin{document}

\maketitle

\clearpage

\tableofcontents

\clearpage

\section{Elektrische Länge}
\[ \ell_\lambda = \frac{\text{physikalische Länge}}{\text{Wellenlänge}} \left\{
\begin{array}{lcl}
\leq \frac{1}{20} & \to & \text{klassische Elektrotechnik} \\
\leq \frac{1}{20} & \to & \text{Wellenphänomene} \\
                  &     & \text{$\Rightarrow$ Feldtheorie} \\
                  &     & \text{$\Rightarrow$ Leitungstheorie} \\
\end{array}
\right. \]

\section{Leitungstheorie}
\begin{itemize}
    \item Längshomogen \\
        Material und Geometrie bleiben erhalten
    \item eine Laufkoordinate Z (1 Dimensional)
    \item $d << \lambda$
\end{itemize}

Ersatzschaltbild: 
\[ R' \left[\frac{\Omega}{m}\right] \]
\[ G' \left[\frac{S}{m}\right] \]
\[ L' \left[\frac{H}{m}\right] \]
\[ C' \left[\frac{F}{m}\right] \]

\[ x(t) = A (\cos(\omega t) + j \cdot \sin(\omega t)) = Re(A \cdot e^{j \omega t}) \]
\[ \frac{\partial^2 \cdot  U(z, t)}{dz^2} - \gamma^2 \cdot U(z, t) = 0 \]
\[ \frac{\partial^2 \cdot  I(z, t)}{dz^2} - \gamma^2 \cdot I(z, t) = 0 \]
Allgemeine Lösung: 
\[ 
    U(z, t) 
    = \underbrace{U_V \cdot e^{-\gamma z} \cdot e^{j \omega t}}_{rechtslaufend} 
    - \underbrace{U_R \cdot e^{+\gamma z} \cdot e^{j \omega t}}_{linkslaufend} 
    = Re(U(z, t)) 
    = U_V \cdot e^{-\alpha z} \cdot \cos(\omega t - \beta z) 
\]

\subsection{Komplexer Ausbreitungskoeffizient}
\[ 
    \gamma 
    = \underbrace{\alpha}_{\text{Dämpfungsbelag }\left[\frac{NP}{m}\right]} 
    + \underbrace{j \beta}_{\text{Phasenbelag }\left[\frac{rad}{m}\right]}
    = \sqrt{(R' + j \omega C') \cdot (G' + j \omega C')}
\]
\[ \alpha_{dB/m} = \alpha_{NP/m} \cdot 8.686 \]

\subsection{Charakteristische Leitungsimpedanz}
\[ Z_0 = \sqrt{\frac{R' + j \omega L'}{G' + j \omega C'}} \]

\subsection{Verlustlose Leitung ($R' = G' = 0$)}
\[ \gamma = j \beta = j \omega \sqrt{L' \cdot C'} \]
\[ \alpha = 0 \]
\[ Z_0 = \sqrt{\frac{L'}{C'}} \]



\end{document}
