\documentclass[a4paper, 10pt, fleqn]{article}

\usepackage{layout}

\title{TKOM - Übertragungstechnik}
\author{daniw}
\date{\today}

\begin{document}

\maketitle

\clearpage

\tableofcontents

\clearpage

\section{Elektrische Länge}
\[ \ell_\lambda = \frac{\text{physikalische Länge}}{\text{Wellenlänge}} \left\{
\begin{array}{lcl}
\leq \frac{1}{20} & \to & \text{klassische Elektrotechnik} \\
\leq \frac{1}{20} & \to & \text{Wellenphänomene} \\
                  &     & \text{$\Rightarrow$ Feldtheorie} \\
                  &     & \text{$\Rightarrow$ Leitungstheorie} \\
\end{array}
\right. \]

\section{Leitungstheorie}
\begin{itemize}
    \item Längshomogen \\
        Material und Geometrie bleiben erhalten
    \item eine Laufkoordinate Z (1 Dimensional)
    \item $d << \lambda$
\end{itemize}

Ersatzschaltbild: 
\[ R' \left[\frac{\Omega}{m}\right] \]
\[ G' \left[\frac{S}{m}\right] \]
\[ L' \left[\frac{H}{m}\right] \]
\[ C' \left[\frac{F}{m}\right] \]

\[ x(t) = A (\cos(\omega t) + j \cdot \sin(\omega t)) = Re(A \cdot e^{j \omega t}) \]
\[ \frac{\partial^2 \cdot  U(z, t)}{dz^2} - \gamma^2 \cdot U(z, t) = 0 \]
\[ \frac{\partial^2 \cdot  I(z, t)}{dz^2} - \gamma^2 \cdot I(z, t) = 0 \]
Allgemeine Lösung: 
\[ 
    U(z, t) 
    = \underbrace{U_V \cdot e^{-\gamma z} \cdot e^{j \omega t}}_{rechtslaufend} 
    - \underbrace{U_R \cdot e^{+\gamma z} \cdot e^{j \omega t}}_{linkslaufend} 
    = Re(U(z, t)) 
    = U_V \cdot e^{-\alpha z} \cdot \cos(\omega t - \beta z) 
\]

\subsection{Komplexer Ausbreitungskoeffizient}
\[ 
    \gamma 
    = \underbrace{\alpha}_{\text{Dämpfungsbelag }\left[\frac{NP}{m}\right]} 
    + \underbrace{j \beta}_{\text{Phasenbelag }\left[\frac{rad}{m}\right]}
    = \sqrt{(R' + j \omega C') \cdot (G' + j \omega C')}
\]
\[ \alpha_{dB/m} = \alpha_{NP/m} \cdot 8.686 \]

\subsection{Charakteristische Leitungsimpedanz}
\[ Z_0 = \sqrt{\frac{R' + j \omega L'}{G' + j \omega C'}} \]

\subsection{Verlustlose Leitung ($R' = G' = 0$)}
\[ \gamma = j \beta = j \omega \sqrt{L' \cdot C'} \]
\[ \alpha = 0 \]
\[ Z_0 = \sqrt{\frac{L'}{C'}} \]

\section{Reflektion}
\[ Z_0 = \sqrt{\frac{R' + j \omega L'}{G' + j \omega C'}} \]
\[ \gamma = \alpha + j \beta = \sqrt{(R' + j \omega L') \cdot (G' + j \omega C')} \]

\subsection{Reflektionskoeffizient}
\[ r = \frac{U_R}{U_V} = \frac{R_L - Z_0}{R_L + Z_0} = -\frac{I_R}{I_V} \]
\[ U_R =  r \cdot U_V \qquad | Z = \ell \]
\[ I_R = -r \cdot I_V \qquad | Z = \ell \]
Fall $R_L = 0$ $r = -1$
\[ U_R = -U_V \]
\[ \sum\limits_i U_i = U_V + U_R = U_V - U_V = 0 \]
\[ \sum\limits_i I_i = I_V + I_R = I_V + I_V = 2 \cdot I_V \]
\[ P_L = 0 \quad \Rightarrow P_R = P_V \]
Fall $R_L = \infty$ $r = +1$
\[ U_R = -U_V \]
\[ \sum\limits_i U_i = U_V + U_R = U_V + U_V = 2 \cdot U_V \]
\[ \sum\limits_i I_i = I_V + I_R = I_V - I_V = 0 \]
\[ P_L = 0 \quad \Rightarrow P_R = P_V \]
Wertebereich von $r$: 
\[ |r^2| \leq 1 \]

\subsection{Wellenanpassung (Impedanzanpassung)}
\[ Z_L = Z_0 \]

\subsection{Leistungsanpassung}
\[ Z_L = {Z_0}^* \]

\section{Drahtlose Übertragung}

\subsection{Linkbudget Analyse}
\begin{zebratabular}{ll}
    \rowcolor{gray} Komponente &
        Wert \\
    Sendeleistung $P_{TX}$ &
        $+30\si{\decibel}_m$ \\
    Leitung, Anpassung $a_t$ &
        $-3\si{\decibel}$ \\
    Antennengewinn &
        $+10\si{\decibel}_i$ \\
    Kanaldämpfung &
        $-100\si{\decibel}$ \\
    Antenne &
        $+10\si{\decibel}_i$ \\
    Polarisationsverluste &
        $-3\si{\decibel}$ \\
    Leistung, Anpassung &
        $-6\si{\decibel}$ \\
    Empfangsleistung $P_{RX}$ &
        $-62\si{\decibel}_m$ \\
    Link Margin &
        $-8\si{\decibel}$ \\
    Empfängerempfindlichkeit &
        $-70\si{\decibel}_m$ \\
\end{zebratabular}

\subsection{Fernfeld-Kriterium}
\[ d_{min} = \frac{2 \cdot D^2}{\lambda} \]

\subsection{Antennenleistung}
\[ P_{ant} = P \cdot (1 - |r|^2) \]

\subsection{Welligkeitsfaktor}
\[ s = \frac{1 + |r|}{1 - |r|} \]

\subsection{Wirkungsgrad}
\[ \eta = \frac{R_{rad}}{R_v + R_{rad}} \]

\subsection{Richtwirkungsfaktor $D$}
\[ D = \frac{S_{max}(\theta, \varphi) \si{\watt\per\square\metre}}
    {S_{iso} \si{\watt\per\square\metre}} \]
\[ D = \frac{4 \pi}{\varphi_{-3dB}(rad) + \theta_{-3dB}(rad)} \]
\[ D = \frac{S_r}{\frac{P_{rad}}{4 \cdot \pi \cdot R^2}} \]

\subsection{Antennengewinn}
\[ G = n \cdot D \]
\[ G_{dB} = 10 \cdot \log_{10}(G) \]

\subsection{Effective Radiated Power}
\[ ERP = P_{ant} \cdot G \]
WLAN: 100mW

\subsection{Empfangsleistung}
\[ S_r (\si{\watt\per\square\metre}) \]
\[ P_r = S_r \cdot A_e \]
\[ A_e = \frac{\lambda^2}{4 \pi} \cdot G \]
\[ P_r = \frac{E^2}{n_0} \cdot A_e \]
Wellenwiderstand $n_0 = \sqrt{\frac{\mu_0}{\epsilon_0}} 
= \frac{|E|}{|H|} \approx 377\si{\ohm}$

\subsection{Freiraumdämpfung}
\[ P_{RX} = P_{TX} \cdot a = P_{TX} \left(\frac{\lambda}{4 \pi \cdot d}\right)^2 \]

\subsection{Kanalmodell}
\[ P_r = P_t \cdot K \cdot \left(\frac{d_0}{d}\right)^\gamma \]
\[ d_0 = 1\si{\metre} \]
\[ \gamma: \text{ Parameter für Umfeld} \]
\[ K = \left(\frac{\lambda}{4 \pi \cdot d_0}\right) \cdot G_r \cdot G_t \]

\subsection{Grenzwert für nichtionisierende Strahlung}
4\si{\volt\per\metre}

\subsection{Reflektionskoeffizient}
\[ \Gamma_E = \frac{\eta_2 - \eta_1}{\eta_2 + \eta_1} = -\Gamma_H \]
\[ \eta = \sqrt{\frac{\hat{z}}{\hat{y}}} \]
Freiraum: 
\[ \hat{z} = j \omega\mu_0 \]
\[ \hat{y} = j \omega \varepsilon_0 \]
Nichtmagnetische Leiter:
\[ \hat{z} = j \omega\mu_0 \]
\[ \hat{y} = \rho + j \omega \varepsilon_0 \]
\[ E_t = E_e \cdot (1 + \Gamma) \]

\subsection{Absorbtion}
\[ e^{-\frac{x}{\delta}} \]
\[ \delta = \frac{1}{\sqrt{\pi \cdot f \cdot \rho \cdot \mu}} \]

\subsection{Fernfeld}
Schirmeffektivität
\[ SE = - 10 \cdot \log_{10}\left(\frac{P_a}{P_E}\right) \]

\subsubsection{Öffnung}
\[ 2a << \frac{\lambda}{2} \qquad \text{Schirmwirkung gut} \]
\[ 2a > \frac{\lambda}{2} \qquad \text{Schirmwirkung nahezu inexistent} \]
\[ 2a < 0.1 \lambda \quad \rightarrow \quad SE \approx -10 \cdot \log_{10}\left(A_{Löcher}\left(\frac{a}{\lambda}\right)^4\right) \si{\decibel} \]

\section{EMV}
\[ Z_0 = \sqrt{\frac{R' + j \omega L'}{G' + j \omega C'}} \]
\[ G' \approx 0 \]
\[ Z_0 = \sqrt{\frac{R' + j \omega L'}{j  \omega C'}} \]

\subsection{Grenzfrequenz des Kabels}
\[ R' = \omega L' \]

\section{Modulation}
\[ s(t) = \sum\limits_k a_k \cdot p(t-kt) \]
\[ a_k = \{0, 1\} \]
Mögliche Anforderungen an Basisband Modulator
\begin{itemize}
    \item Synchronisation
    \item DC frei
    \item Zusatzinformationen
    \item Spektrumsbedarf optimieren
    \item Zeitmultiplexierung
    \item \ldots
\end{itemize}

\section{Quantisierung}
\subsection{Rauschleistung}
\[ {\delta_N}^2 = \frac{q}{12} \]

\subsection{Serielle Datenrate}
\[  r_b = n \cdot f_{Sample} \]

\subsection{Signal - Rausch Verhältnis}
\[ \left(\frac{S}{N}\right)_{A} = 
10 \cdot \log_{10} \left(\frac{S_A}{{\delta_q}^2 + {\delta_k}^2}\right) \]
\[ \left(\frac{S}{N}\right)_{A_dB} = 
10 \cdot \log_{10} \left(3 \cdot 2^{2 \cdot n} \cdot S_x\right) \leq n \cdot 6 + 4.8 \]
\begin{tabular}{@{}ll}
    ${\delta_q}^2$  & : Quantisierungsrauschen \\
    ${\delta_k}^2$  & : Rauschleistung im Kanal \\
    ${S_x}$         & : normierte Signalleistung \\
\end{tabular}

\end{document}
